\documentclass{report}
\usepackage{amssymb}

\begin{document}

\chapter{Notation}

\section{Functions of Matrices}

Let $t_{i,j}$ denote the element in row $i$ and column $j$ of the matrix $T$.
Let $f=f(T)$ be a scalar function on the set of $n \times n$ real matrices 
$M_n$. \newline

\noindent {\it In the manipulations below it is assumed that $f$ and all of 
its first and second partial derivatives exist and are continuous. Some of the 
TMP metrics do not satisfy this requirement.} For example, the first partial
derivatives of the norm of $T$ to the first power (i.e. $|T|$) do not exist 
when $T=0$. 
They do exist everywhere else in $M_n$, however.  The first partial 
derivatives of 
the absolute value, $|det(T)|$, do not exist when $T$ is singular. Therefore,
special care must be taken for metrics which include these expressions. 
\newline

\section{First Derivatives}

Let $\frac{\partial f}{\partial T}$ be the matrix of partial derivatives of the scalar function $f(T)$ where $T$ is a matrix, such that each element 
$\left(\frac{\partial f}{\partial T}\right)_{i,j} = \frac{\partial f}{\partial \, t_{i,j}}$.

\section{Second Derivatives}

Let $\frac{\partial^2 f}{\partial T_i \partial T_j}$ be the matrix of second partial derivatives with respect to the $i^{th}$ and $j^{th}$ rows of $T$:

\begin{equation}
\frac{\partial^2 f}{\partial T_i \partial T_j} = 
\left[ \begin{array}{cccc}
\frac{\partial^2 f}{\partial t_{i,1} \partial t_{j,1}} &
\frac{\partial^2 f}{\partial t_{i,1} \partial t_{j,2}} &
\ldots &
\frac{\partial^2 f}{\partial t_{i,1} \partial t_{j,n}} \\
\frac{\partial^2 f}{\partial t_{i,2} \partial t_{j,1}} &
\frac{\partial^2 f}{\partial t_{i,2} \partial t_{j,2}} \\
\vdots & & \ddots \\
\frac{\partial^2 f}{\partial t_{i,n} \partial t_{j,1}} & & &
\frac{\partial^2 f}{\partial t_{i,n} \partial t_{j,n}}
\end{array} \right]
\end{equation}

and

\begin{equation}
\label{dsecelem}
\left(\frac{\partial^2 f}{\partial T_i \partial T_j}\right)_{k,l} = 
\frac{\partial^2 f}{\partial t_{i,k} \partial t_{j,l}}
\end{equation}

Thus the symmetric Hessian of $f(T)$, with the elements of $T$ occuring in row-major order, is composed of blocks of the form $\frac{\partial^2 f}{\partial T_i \partial T_j}$:

\begin{eqnarray}
\nabla^2 f(T) & = &
\left[ \begin{array}{c|c|c}
\frac{\partial^2 f}{\partial T_1 \partial T_1} &
\ldots &
\frac{\partial^2 f}{\partial T_1 \partial T_n} \\
\hline
\vdots & \ddots \\
\hline
\frac{\partial^2 f}{\partial T_1 \partial T_n} & 
\ldots & 
\frac{\partial^2 f}{\partial T_n \partial T_n} \\
\end{array} \right]
\\ & = &
\left[ \begin{array}{ccc|c|ccc}
\frac{\partial^2 f}{\partial t_{1,1}^2} &
\ldots &
\frac{\partial^2 f}{\partial t_{1,1} \partial t_{1,n}} &
&
\frac{\partial^2 f}{\partial t_{1,1} \partial t_{n,1}} &
\ldots &
\frac{\partial^2 f}{\partial t_{1,1} \partial t_{n,n}} \\
\vdots & \ddots & \vdots &
\ldots &
\vdots & \ddots & \vdots \\
\frac{\partial^2 f}{\partial t_{1,n} \partial t_{1,1}} &
\ldots &
\frac{\partial^2 f}{\partial t_{1,n}^2} &
&
\frac{\partial^2 f}{\partial t_{1,n} \partial t_{n,1}} &
\ldots &
\frac{\partial^2 f}{\partial t_{1,n} \partial t_{n,n}} \\
\hline
& \vdots & & \ddots & & \vdots \\
\hline
\frac{\partial^2 f}{\partial t_{n,1} \partial t_{1,1}} &
\ldots &
\frac{\partial^2 f}{\partial t_{n,1} \partial t_{1,n}} &
&
\frac{\partial^2 f}{\partial t_{n,1}^2} &
\ldots &
\frac{\partial^2 f}{\partial t_{n,1} \partial t_{n,n}} \\
\vdots & \ddots & \vdots &
\ldots &
\vdots & \ddots & \vdots \\
\frac{\partial^2 f}{\partial t_{n,n} \partial t_{1,1}} &
\ldots &
\frac{\partial^2 f}{\partial t_{n,n} \partial t_{1,n}} &
&
\frac{\partial^2 f}{\partial t_{n,n} \partial t_{n,1}} &
\ldots &
\frac{\partial^2 f}{\partial t_{n,n}^2} 
\end{array} \right] \nonumber \\
\end{eqnarray}

\chapter{Basic Matrix Derivatives}


\section{Chain rule for matrix multiplication}

\subsection{First Derivative}

The following demonstrates that for the product of square matrices $T = A Z$,
that $\frac{\partial f(T)}{\partial A} = \frac{\partial f(T)}{\partial T} Z^t$. \newline

\noindent Beginning with the chain rule:

\begin{equation} 
\label{eqn0}
\frac{\partial f}{\partial a_{i,j}} = 
\sum_{l=1}^{n} \sum_{k=1}^{n} 
\frac{\partial f}{\partial t_{l,k}} 
\frac{\partial t_{l,k}}{\partial a_{i,j}}
\end{equation}

\noindent Since $T=AZ$, matrix multiplication gives:

\begin{equation}
\label{eqn1}
t_{l,k} = \sum_{r=1}^n a_{l,r} z_{r,k}
\end{equation}

\noindent Taking the first derivative of Equation \ref{eqn1}:

\begin{eqnarray}
%\label{eqn2}
\frac{\partial t_{l,k}}{\partial a_{i,j}} 
  &=& \sum_{r=1}^n \delta_i^l \delta_j^r z_{r,k} \\
  &=& \delta_i^l z_{j,k} \label{eqn2}
\end{eqnarray}

\noindent Substituting Equation \ref{eqn2} into \ref{eqn0}:

\begin{eqnarray}
\frac{\partial f}{\partial a_{i,j}} &=& \sum_{l=1}^n \sum_{k=1}^n
 \frac{\partial f}{\partial t_{l,k}} \delta_i^l z_{j,k} \\
&=& \sum_{k=1}^{n} \frac{ \partial f}{ \partial t_{i,k}} z_{j,k} \label{eqn4}
\end{eqnarray}

\noindent which is:

\begin{equation}
\frac{\partial f}{\partial A} = \frac{\partial f}{\partial T} Z^t
\end{equation}


\subsection{Second Derivative}

Define:

\begin{equation}
C = \frac{\partial^2 f}{\partial A_i \partial A_j}
\end{equation}

\noindent for some $i$ and $j$ such that from Equation \ref{dsecelem}:

\begin{equation}
\label{chainsecondelem}
c_{k,l} = \frac{\partial^2 f}{\partial a_{i,k} \partial a_{j,l}}
\end{equation}

\noindent Substituting Equation \ref{eqn4} into Equation \ref{chainsecondelem}:

\begin{eqnarray}
c_{k,l} &=& \frac{\partial}{\partial a_{i,k}} \sum_{p=1}^n \frac{ \partial f}{\partial t_{j,p}} z_{l,p} \\
        &=& \sum_{p=1}^n \frac{\partial}{\partial t_{j,p}} \frac{\partial f}{\partial a_{i,k}} z_{l,p} \\
\label{chainsecondr1}
        &=& \sum_{p=1}^n \sum_{q=1}^n \frac{\partial^2 f}{ \partial t_{j,p} \partial t_{i,q}} z_{l,p} z_{k,q} 
\end{eqnarray}

\noindent Let:

\begin{equation}
D = \frac{\partial^2 f}{\partial T_i \partial T_j}
\end{equation}
which, given the definition in Equation \ref{dsecelem}:
\begin{equation}
d_{q,p} = \frac{\partial^2 f}{\partial t_{i,q} \partial t_{j,p}}
\end{equation}

\noindent Define matrices $X$ and $Y$:

\begin{eqnarray}
X &=& Z D \\
Y &=& X Z^t = Z D Z^t
\end{eqnarray}

\noindent The elements of $X$ and $Y$ are then:

\begin{eqnarray}
x_{k,p} & = & \sum_{q=1}^n z_{k,q} d_{q,p} \\
y_{k,l} & = & \sum_{p=1}^n x_{k,p} z_{l,p} \\ 
        & = & \sum_{p=1}^n \sum_{q=1}^n z_{k,q} d_{q,p} z_{l,p} \\
\label{chainsecondr2}
        & = & \sum_{p=1}^n \sum_{q=1}^n \frac{\partial^2 f}{\partial t_{j,p} \partial t_{i,q}} z_{l,p} z_{k,q}
\end{eqnarray}

\noindent Equation \ref{chainsecondr1} is equal to Equation \ref{chainsecondr2}, so:

\begin{eqnarray}
\frac{\partial^2 f}{\partial A_i \partial A_j} &=& C \\
&=& Y \\
&=& Z D Z^t \\
&=& Z \frac{\partial^2 f}{\partial T_i \partial T_j} Z^t 
\end{eqnarray}


\section{Derivatives of Basic TMP Scalar Functions}
All TMP metrics are functions of the matrices $A$, $Z$, or $T=AZ$.
Each such metric in turn is some combination of the basic scalar functions
$|T|^m$, $det(T)$, $tr(T)$, $|T^t T|$, or $|adj T|^2$.  Therefore, derivatives
of these basic TMP functions will be derived first. \newline

\subsection{Derivatives of $tr(T)$}

\begin{eqnarray}
f(T) & = & tr(T) = \sum_{k=1}^n t_{k,k} \\
\frac{\partial f}{\partial t_{i,j}} &=& \delta_i^j \\
\frac{\partial f}{\partial T} &=& I 
\end{eqnarray}

\noindent All of the second derivatives are zero. \newline

\subsection{Derivatives of $|T|^2$}

\begin{eqnarray}
f(T) &=& |T|^2 = \sum_{i=1}^n \sum_{j=1}^n t_{i,j}^2 \\
\label{tsquaredterm}
\frac{\partial f}{\partial t_{i,j}} &=& 2 t_{i,j} \\
\frac{\partial f}{\partial T} &=& 2 T
\end{eqnarray}

\noindent Beginning with Equation \ref{tsquaredterm}:
\begin{eqnarray}
\frac{\partial^2 f}{\partial t_{i,j} \partial t_{k,l}} &=& 2 \delta_i^k \delta_j^l \\
\left(\frac{\partial^2 f}{\partial T_i \partial T_j}\right)_{r,s} &=& 2 \delta_i^j \delta_r^s \\
\frac{\partial^2 f}{\partial T_i \partial T_j} &=& 2 \delta_i^j I
\end{eqnarray}

\noindent All non-diagonal blocks of the Hessian are zero due to the $\delta_i^j$.  Thus, if $I_{n^2}$ is the $n^2 \times n^2$ identity matrix, then:

\begin{equation}
\frac{\partial^2}{\partial T^2} \, |T|^2 = 2 \, I_{n^2}
\end{equation}

\subsection{Derivatives of $f(T)=|T|^m$}
Using \ref{tsquaredterm},
\begin{eqnarray}
2 T & = & \frac{\partial}{\partial T} |T|^2 \nonumber \\
 & = & 2 |T| \frac{\partial}{\partial T} |T| \nonumber
\end{eqnarray}
so that when $T \neq 0$,
\begin{eqnarray}
\frac{\partial}{\partial T} |T| & = & \frac{T}{|T|}
\end{eqnarray}
Let $m=\pm 1, \pm 2,\pm 3,\ldots$  Then
\begin{eqnarray}
\frac{\partial}{\partial T} |T|^m & = & m |T|^{m-1} \frac{\partial }{\partial T}|T| \\
 & = & m |T|^{m-2} \, T \label{firstdervoftm}
\end{eqnarray}
For $m=2,3,\ldots$, the latter formula is valid even when $T=0$. \newline

\noindent Particularly useful are
\begin{eqnarray}
\frac{\partial}{\partial T} |T|^3 & = & 3 |T| \, T \\
\frac{\partial}{\partial T} |T|^4 & = & 4 |T|^2 \, T 
\end{eqnarray}

\noindent {\bf Second Partial Derivatives of $f(T)=|T|^m$} \newline

\noindent Write \ref{firstdervoftm} in index form as
\begin{eqnarray}
\frac{\partial}{\partial t_{i,j}} |T|^m & = & m |T|^{m-2} \, t_{i,j}
\end{eqnarray}
Then
\begin{eqnarray}
\frac{\partial^2 f}{\partial t_{i,j} \partial t_{k,l}} = m \, \delta_i^k \delta_j^l |T|^{m-2} + m (m-2) \, t_{i,j} t_{k,l} |T|^{m-4}
\end{eqnarray}
and 
\begin{eqnarray}
\frac{\partial^2 f}{\partial T_i \partial T_j} = m \,  |T|^{m-4} \, \left\{ \delta_i^j \, |T|^2 \, I + (m-2) S_{ij} \right\} 
\end{eqnarray}
with $S_{ij}$ an $n \times n$ matrix with entries
\begin{eqnarray}
\left( S_{ij} \right)_{r,s} = t_{i,r} \, t_{j,s}
\end{eqnarray}
When $m=2$ or $m \geq 4$ the second derivatives are valid for all $T \in M_3$;
otherwise, $T=0$ is not allowed. \newline

\subsection{First Partial Derivative of $det(T)$}

\begin{eqnarray}
f(T) &=& det(T) \\
     &=& \sum_{k=1}^n t_{l,k} C_{l,k}(T) \quad \forall \quad 1 \le l \le n \\
\label{det1}
     &=& \sum_{k=1}^n (-1)^{l+k} t_{l,k} M_{l,k}(T) \quad \forall \quad 1 \le l \le n
\end{eqnarray}

\noindent For each term of $\frac{\partial f}{\partial T}$, take the partial derivative of of Equation \ref{det1} for $l = i$ with respect to the corresponding element of $T$:

\begin{eqnarray}
\frac{\partial f}{\partial t_{i,j}} &=& 
\sum_{k=1}^n (-1)^{i+k} \left[
 \frac{\partial t_{i,k}}{\partial t_{i,j}} M_{i,k}(T) + 
 \frac{\partial M_{i,k}}{\partial t_{i,k}} t_{i,k} \right] \\
 &=& \sum_{k=1}^n (-1)^{i+k} \left[
 \delta^k_j M_{i,k}(T) + 
 \frac{\partial M_{i,k}}{\partial t_{i,k}} t_{i,k} \right] \\
\label{det2}
 &=& (-1)^{i+j} M_{i,j}(T) + \sum_{k=1}^n (-1)^{i+k} \frac{\partial 
   M_{i,k}}{\partial t_{i,k}} t_{i,k}
\end{eqnarray}

By definition, the minor for row $i$ and column $j$ contains terms from neither row $i$ nor column $j$. Therefore the partial derivative of a minor with respect to any term from either the corresponding row or column of the original matrix is zero:
\begin{equation}\label{det3}
\frac{\partial M_{i,j}}{\partial t_{i,k}} = 
\frac{\partial M_{i,j}}{\partial t_{k,j}} = 0 \quad \forall \quad k
\end{equation}

Thus all the terms in the summation in Equation \ref{det2} are zero, leaving:

\begin{equation}\label{det_deriv_term}
\frac{\partial f}{\partial t_{i,j}} = (-1)^{i+j}M_{i,j}(T)
\end{equation}

Which is the definition of the cofactor matrix of $T$, and the transpose of the definition of the adjugate of $T$:

\begin{equation}
\frac{\partial f}{\partial T} = (adj(T))^t = adj(T^t)
\end{equation}

\subsection{Second Partial Derivative of $det(T)$}

The partial derivative of Equation \ref{det_deriv_term} is:

\begin{equation}
\frac{\partial^2 f}{\partial t_{i,j} \partial t_{k,l}} = (-1)^{i+j}\frac{\partial M_{i,j}(T)}{\partial t_{k,l}}
\end{equation}

For a $2 \times 2$ matrix, $M_{i,j}(T) = t_{3-i,3-j}$:
\begin{equation}
\frac{\partial^2 f}{\partial t_{i,j} \partial t_{k,l}} = 
(-1)^{i+j} \delta_{3-i}^k \delta_{3-j}^l
\end{equation}

\begin{equation}
\left(\frac{\partial^2 f}{\partial T_i \partial T_j}\right)_{r,s} =
(-1)^{i+r} \delta_{3-i}^j \delta_{3-r}^s
\end{equation}

\begin{equation}
\frac{\partial^2 det(T)}{\partial T^2} =
\left[ \begin{array}{cccc}
0 & 0 & 0 & 1 \\
0 & 0 & -1 & 0 \\
0 & -1 & 0 & 0 \\
1 & 0 & 0 & 0 
\end{array} \right]
\end{equation}

For a $3 \times 3$ matrix, the Hessian of the determinant is:

\begin{equation}
\frac{\partial^2 det(T)}{\partial T^2} =
\left[ \begin{array}{ccccccccc}
 0 & 0 & 0 & 0 & t_{3,3} & -t_{3,2} & 0 & -t_{2,3} & t_{2,2} \\
 0 & 0 & 0 & -t_{3,3} & 0 & t_{3,1} & t_{2,3} & 0 & -t_{2,1} \\
 0 & 0 & 0 & t_{3,2} & -t_{3,1} & 0 & -t_{2,2} & t_{2,1} & 0 \\
 0 & -t_{3,3} & t_{3,2} & 0 & 0 & 0 & 0 & t_{1,3} & -t_{2,1} \\
 t_{3,3} & 0 & -t_{3,1} & 0 & 0 & 0 & -t_{1,3} & 0 & t_{1,1} \\
 -t_{3,2} & t_{3,1} & 0 & 0 & 0 & 0 & t_{1,2} & -t_{1,1} & 0 \\
 0 & t_{2,3} & -t_{2,2} & 0 & -t_{1,3} & t_{1,2} & 0 & 0 & 0 \\
 -t_{2,3} & 0 & t_{2,1} & t_{1,3} & 0 & -t_{1,1} & 0 & 0 & 0 \\
 t_{2,2} & -t_{2,1} & 0 & -t_{1,2} & t_{1,1} & 0 & 0 & 0 & 0
\end{array} \right]
\end{equation}

Each non-diagonal $3 \times 3$ block of the $3 \times 3$ result can be expressed as:

\begin{equation}
\frac{\partial^2 f}{\partial T_i \partial T_j}
 = s\left[\begin{array}{ccc}
0 & t_{k,3} & -t_{k,2} \\
-t_{k,3} & 0 & t_{k,1} \\
t_{k,2} & -t_{k,1} & 0 
\end{array}\right]\end{equation}

where 
\begin{eqnarray}
k &=& (3-i) + (3-j) \\
s &=& (-1)^{i+j}\frac{i-j}{|i-j|}
\end{eqnarray}

Both the $2 \times 2$ and $3 \times 3$ results are composed of skew-symmetric $n \times n$ submatrices.  For the $3 \times 3$ result, the terms in a submatrix are the values from a single row of $T$.  The diagonal submatrices for both results are zero, and the diagonal terms of each submatrix are zero.


% \section{Derivatives of $T^p$}
% 
% Define $T^p$ for any positive integer $p$ to be the product of $p$ instances of the matrix $T$.  Define $T^0 = I$.  The derivative of the first power is obvious:
% \begin{eqnarray}
% T^1 &=& T \\
% (T^1)_{i,j} &=& t_{i,j} \\
% \frac{\partial (T^1)_{i,j}}{\partial t_{r,s}} &=& \delta^i_r \delta^j_s \\
% \frac{\partial T^1}{\partial t_{r,s}} &=& E_{r,s}
% \end{eqnarray}
% 
% For the second power of $T$:
% 
% \begin{eqnarray}
% T^2 &=& T T \\
% (T^2)_{i,j} &=& \sum_{k=1}^n t_{i,k} t_{k,j} \\
% \frac{\partial (T^2)_{i,j}}{\partial t_{r,s}} &=& 
%   \sum_{k=1}^n \left( \delta^i_r \delta^k_s t_{k,j} 
%   + t_{i,k} \delta^k_R \delta^j_s 
%   \right) \\
% \frac{\partial T^2}{\partial t_{r,s}} &=& \frac{\partial T}{\partial t_{r,s}} T 
%   + T \frac{\partial T}{\partial t_{r,s}} 
% \end{eqnarray}
% 
% And for the third power of $T$:
% 
% \begin{eqnarray}
% T^3 &=& T T T\\
% (T^3)_{i,j} &=& \sum_{p=1}^n \sum_{k=1}^n t_{i,k} t_{k,p} t_{p,j} \\
% \frac{\partial (T^3)_{i,j}}{\partial t_{r,s}} &=& 
%   \sum_{p=1}^n \sum_{k=1}^n \left( 
%   \delta^i_r \delta^k_s t_{k,p} t_{p,j} 
%   + t_{i,k} \delta^k_r \delta^p_s t_{p,j} 
%   + t_{i,k} t_{k,p} \delta^p_r \delta^j_s 
%   \right)\\
% \frac{\partial T^3}{\partial t_{r,s}} &=& \frac{\partial T}{\partial t_{r,s}} T T + T \frac{\partial T}{\partial t_{r,s}} T + T T \frac{\partial T}{\partial t_{r,s}} 
% \end{eqnarray}
% 
% And for the fourth power of $T$:
% 
% \begin{eqnarray}
% T^4 &=& T T T T\\
% (T^4)_{i,j} &=& \sum_{q=1}^n \sum_{p=1}^n \sum_{k=1}^n t_{i,k} t_{k,p} t_{p,q} t_{q,j} \\
% \nonumber 
% \frac{\partial (T^4)_{i,j}}{\partial t_{r,s}} &=& 
%   \sum_{q=1}^n \sum_{p=1}^n \sum_{k=1}^n ( 
%    \delta^i_r \delta^k_s t_{k,p} t_{p,q} t_{q,j} 
%    + t_{i,k} \delta^k_r \delta^p_s t_{p,q} t_{q,j} \\ & & {} 
%    + t_{i,k} t_{k,p} \delta^p_r \delta^q_s t_{q,j} 
%    + t_{i,k} t_{k,p} t_{p,q} \delta^q_r \delta^j_s)\\
% \frac{\partial T^4}{\partial t_{r,s}} &=& 
%   \frac{\partial T}{\partial t_{r,s}} T T T 
%   + T \frac{\partial T}{\partial t_{r,s}} T T 
%   + T T \frac{\partial T}{\partial t_{r,s}} T 
%   + T T T \frac{\partial T}{\partial t_{r,s}}
% \end{eqnarray}
% 
% I'm not sure how to formalize the next step, but the pattern is clear:
% 
% \begin{equation}
% \frac{\partial T^p}{\partial t_{r,s}}
%  = \sum_{i=1}^p\left( T^{i-1} \frac{\partial T}{\partial t_{r,s}} T^{p-i} 
%    \right) \quad \forall \quad p \in \mathbb{Z}, p > 0
% \end{equation}
% 
% Going back to the formulas for single terms in the derivatives, we can simplify further: several of the summations can be eliminated because all but one 
% of the terms are multiplied by a Kronecker delta equal to zero:
% \begin{eqnarray}
% \frac{\partial (T^1)_{i,j}}{\partial t_{r,s}} &=& \delta^i_r \delta^j_s \\
% \frac{\partial (T^2)_{i,j}}{\partial t_{r,s}} &=& \delta^i_r t_{s,j} + t_{i,r} \delta^j_s \\
% \frac{\partial (T^3)_{i,j}}{\partial t_{r,s}} &=& 
% \delta_r^i \sum_{p=1}^n t_{s,p} t_{p,j} + t_{i,r} t_{s,j} + 
% \delta_s^j \sum_{k=1}^n t_{i,k} t_{k,r} \\
%  & = & \delta_r^i (T^2)^{s,j} + t_{i,r} t_{s,j}
%  + \delta_s^j (T^2)_{i,r} \\
% \nonumber
% \frac{\partial (T^4)_{i,j}}{\partial t_{r,s}} &=& 
% \delta^i_r \sum_{q=1}^n \sum_{p=1}^n t_{s,p} t_{p,q} t_{q,j} +
% t_{i,r} \sum_{q=1}^n t_{s,q} t_{q,j} \\ & & {} +
% t_{s,j} \sum_{k=1}^n t_{i,k} t_{k,r} +
% \delta^j_s \sum_{p=1}^n \sum_{k=1}^n t_{i,k} t_{k,p} t_{p,r} \\
%  & = & \delta^i_r (T^3)_{s,j} +
% t_{i,r} (T^2)_{s,j} + 
% t_{s,j} (T^2)_{i,r} +
% \delta^j_s (T^3)_{i,r}
% \end{eqnarray}
% 
% The pattern isn't evident yet, so for $T^5$:
% 
% \begin{eqnarray}{ccc}
% (T^5)_{i,j} &=& \sum_{u=1}^n \sum_{q=1}^n \sum_{p=1}^n \sum_{k=1}^n
% t_{i,k} t_{k,p} t_{p,q} t_{q,u} t_{u,j} \\
% \nonumber
% \frac{\partial (T^5)_{i,j}}{\partial t_{r,s}} &=&
%   \sum_{u=1}^n \sum_{q=1}^n \sum_{p=1}^n \sum_{k=1}^n (
%   \delta^i_r \delta^k_s t_{k,p} t_{p,q} t_{q,u} t_{u,j} \\ \nonumber & & {} +
%   t_{i,k} \delta^k_r \delta^p_s t_{p,q} t_{q,u} t_{u,j} +
%   t_{i,k} t_{k,p} \delta^p_r \delta^q_s t_{q,u} t_{u,j} \\ & & {} +
%   t_{i,k} t_{k,p} t_{p,q} \delta^q_r \delta^u_s t_{u,j} +
%   t_{i,k} t_{k,p} t_{p,q} t_{q,u} \delta^u_r \delta^j_s )\\
% \nonumber
% &=& \delta^i_r \sum_{u=1}^n \sum_{q=1}^n \sum_{p=1}^n t_{s,p} t_{p,q} t_{q,u} t_{u,j}
%  + t_{i,r} \sum_{u=1}^n \sum_{q=1}^n t_{s,q} t_{q,u} t_{u,j}
%  \\ \nonumber & & {} + 
%  \sum_{u=1}^n \sum_{k=1}^n t_{i,k} t_{k,r} t_{s,u} t_{u,j}
%  + t_{s,j} \sum_{p=1}^n \sum_{k=1}^n t_{i,k} t_{k,p} t_{p,r}
%  \\ & & {} + 
%  \delta^j_s \sum_{q=1}^n \sum_{p=1}^n \sum_{k=1}^n t_{i,k} t_{k,p} t_{p,q} t_{q,r} \\
% \nonumber
% &=& \delta^i_r (T^4)_{s,j} + t_{i,r} (T^3)_{s,j} + (T^2)_{i,r}(T^2)_{s,j}
%   + t_{s,j} (T^3)_{i,r} + \delta^j_s (T^4)_{i,r} \\
% \end{eqnarray}


\chapter{Derivatives of target metrics}

\section{Derivatives of terms $3 \times 2$ $A^{\prime\prime}$ with respect to $2 \times 2$ $A$}

Let $W_{*,i}$ denote the $i^{th}$ column of the matrix $W$.

\begin{eqnarray}
n^\prime_w &=& \frac{W^\prime_{*,1} \times W^\prime_{*,2}}
                   {|W^\prime_{*,1} \times W^\prime_{*,2}|} \\
Z &=& \left[ \frac{W^\prime_{*,1}}{|W^\prime_{*,1}|}, 
              n^\prime_w \times \frac{W^\prime_{*,1}}{|W^\prime_{*,1}|} \right] \\
W &=& Z^t \times W^\prime \\
n^{\prime\prime} &=& \frac{A^{\prime\prime}_{*,1} \times A^{\prime\prime}_{*,2}}
                       {|A^{\prime\prime}_{*,1} \times A^{\prime\prime}_{*,2}|} \\
n_r &=& \left\{ \begin{array}{ll} 
         n^\prime_w & \textrm{if $(n^{\prime\prime} \cdot n^\prime_w) >= 0$} \\
        -n^\prime_w & \textrm{if $(n^{\prime\prime} \cdot n^\prime_w) < 0$}
        \end{array} \right. \\
v &=& \frac{n_r \times n^{\prime\prime}}{|n_r \times n^{\prime\prime}|} \\
R &=& \left[ v, n^{\prime\prime}, v \times n^{\prime\prime} \right] \\
S &=& \left[ v, n_r, v \times n_r \right] \\
T &=& S \times R^t \\
A^\prime &=& T \times A^{\prime\prime} \\
A &=& Z^t \times A^\prime
\end{eqnarray}

\section{Derivatives of $|T-B|^2$}

\begin{equation}
f(T) = |T-B|^2 = \sum_{i=1}^n \sum_{j=1}^n (t_{i,j} - b_{i,j})^2
\end{equation}

\begin{equation}
\frac{\partial f}{\partial t_{i,j}} = 2 (t_{i,j} - b_{i,j})
\end{equation}

\begin{equation}
\frac{\partial f}{\partial T} = 2(T - B)
\end{equation}


\begin{equation}
\frac{\partial^2 f}{\partial t_{i,j} \partial t_{k,l}} = 2 \delta^i_k \delta^j_l \end{equation}

\begin{equation}
\frac{\partial^2 f}{\partial T^2} = 2 I
\end{equation}


\section{Derivatives of Inverse Mean Ratio}

\begin{eqnarray}
f(T) & = & \frac{|T|^2}{n det^{2/n}(T)} \\
g(T) & = & |T|^2 \\
h(T) & = & det(T) \\
e(T) & = & n \cdot h(T)^{2/n} \\
f(T) & = & \frac{g(T)}{e(T)} 
\end{eqnarray}

\begin{eqnarray}
\frac{\partial f}{\partial T} &=& \frac{
\frac{\partial g}{\partial T} e(T) - 
\frac{\partial e}{\partial h} \frac{\partial h}{\partial t} g(T)}
{e^2(T)} \\
&=& 
\frac{2 T e(T) - 
2 h(T)^\frac{2-n}{n} adj(T^t) g(T)}
{e^2(T)} \\
& = &  
\frac{2 T}{e(T)} -
\frac{2 h(T)^{\frac{2-n}{n}} adj^t(T) |T|^2}{e^2(T)} \\
& = &  
\frac{2 T}{n \cdot det(T)^{2/n}} -
\frac{2 det(T)^{\frac{2-n}{n}} adj^t(T) |T|^2}{n^2 \cdot det(T)^{4/n}} \\
& = &  
\frac{2}{n \cdot det^{2/n}(T)}\left[ T - \frac{|T|^2}{n \cdot det(T)} adj(T^t) \right]
\end{eqnarray}

Begin the second derivative with the  expression for individual elements of the first derivative:

\begin{eqnarray}
\frac{\partial f}{\partial t_{i,j}} & = & 
\frac{2}{n \cdot det^{2/n}(T)}\left[ t_{i,j} - \frac{|T|^2}{n \cdot det(T)} adj(T^t)_{i,j} \right] \\
 & = & 
\frac{2}{n}det^{-2/n}(T)\left[ t_{i,j} - \frac{1}{n}|T|^2 det^{-1}(T) adj(T^t)_{i,j}\right] \\
 & = &
\frac{2}{n} t_{i,j} det^{-2/n}(T) - \frac{2}{n^2} |T|^2 det^{\frac{-2}{n} - 1}(T) adj(T^t)_{i,j} 
\end{eqnarray}

And differentiate with respect to an element of $T$:

\begin{eqnarray}
\nonumber
\frac{\partial^2 f}{\partial t_{i,j} \partial t_{k,l}} & = &
\frac{2}{n} \delta_i^k \delta_j^l det^{-2/n}(T) 
  - \frac{4}{n^2} t_{i,j} det^{\frac{-2}{n}-1}(T) adj(T^t)_{k,l} \\
\nonumber
  & & {} - \frac{4}{n^2}t_{k,l} det^{\frac{-2}{n} - 1}(T) adj(T^t)_{i,j} \\
\nonumber
  & & {} + \frac{4+2n}{n^3} |T|^2 det^{\frac{-2}{n} - 2}(T) adj(T^t)_{i,j} adj(T^t)_{k,l} \\
  & & {} - \frac{2}{n^2} |T|^2 det^{\frac{-2}{n} - 1}(T) \frac{\partial adj(T^t)_{i,j}}{\partial t_{k,l}} \\
 & = & \frac{2}{n \cdot det^{2/n}(T)} \delta_i^k \delta_j^l
     + \frac{4 + 2n}{n^3 det^{\frac{2}{n}+2}(T)} |T|^2 adj(T^t)_{i,j} adj(T^t)_{k,l} \\
\nonumber
& & {} - \frac{2}{n^2 det^{\frac{2}{n}+1}} \left[ 
     2 t_{k,l} adj(T^t)_{i,j} + 2 t_{i,j} adj(T^t)_{k,l}
     + |T|^2 \frac{\partial adj(T^t)_{i,j}}{\partial t_{k,l}} \right] 
\end{eqnarray}

Returning to the Hessian notation for the second derivatives from Equation \ref{dsecelem}:


\begin{eqnarray}
\left(\frac{\partial^2 f}{\partial T_i \partial T_j}\right)_{r,s} &=&
\frac{2}{n \cdot det^{2/n}(T)} \delta_i^j \delta_r^s
     + \frac{4 + 2n}{n^3 det^{\frac{2}{n}+2}(T)} |T|^2 adj(T^t)_{i,r} adj(T^t)_{j,s} \\
\nonumber
& & {} - \frac{2}{n^2 det^{\frac{2}{n}+1}} \left[ 
     2 t_{j,s} adj(T^t)_{i,r} + 2 t_{i,r} adj(T^t)_{j,s}
     + |T|^2 \frac{\partial adj(T^t)_{i,r}}{\partial t_{j,s}} \right] 
\end{eqnarray}

Define $A = adj(T^t)$ and $A_i$ equal to the $i^th$ row of $A$.

\begin{eqnarray}
\frac{\partial^2 f}{\partial T_i \partial T_j}
&=& \frac{2 \delta_i^j}{n \cdot det^{2/n}(T)} I
+ \frac{4 + 2n}{n^3 det^{\frac{2}{n}+2}(T)} |T|^2 A_i^t A_j \\ \nonumber
& & {} - \frac{2}{n^2 det^{\frac{2}{n}+1}} \left[ 
2 A_i^t T_j + 2 T_i^t A_j + |T|^2 \frac{\partial^2 det(T)}{\partial T_i \partial T_j} \right]
\end{eqnarray}

Or if, instead, $A = adj(T)$, then

\begin{eqnarray} \label{imrblockhess}
\frac{\partial^2 f}{\partial T_i \partial T_j} 
&=& \frac{2 \delta_i^j}{n \cdot det^{2/n}(T)} I
+ \frac{4 + 2n}{n^3 det^{\frac{2}{n}+2}(T)} |T|^2 A_j^t A_i \\ \nonumber
& & {} - \frac{2}{n^2 det^{\frac{2}{n}+1}} \left[ 
2 T_j^t A_i + 2 A_j^t T_i + |T|^2 \frac{\partial^2 det(T)}{\partial T_i \partial T_j} \right]
\end{eqnarray}



For $2 \times 2$ target matrices, Equation \ref{imrblockhess} simplifies to:
\begin{eqnarray}
\frac{\partial^2 f}{\partial T_i \partial T_j}
&=& \frac{\delta^i_j}{det(T)}I + \frac{|T|^2}{det^3(T)} A_j^t A_i \\ \nonumber
& & {} - \frac{1}{det^2(T)} \left[ T_j^t A_i + A_j^t T_i 
+ \frac{|T|^2}{2} \frac{\partial^2 det(T)}{\partial T_i \partial T_j} \right]
\end{eqnarray}

\end{document}
